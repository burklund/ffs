\pdfoutput=1\relax
\documentclass[reqno]{amsart}
\usepackage{lmodern}
\usepackage[T1]{fontenc}
\usepackage[latin9]{inputenc}

%----------------------------packages----------------------------------%

\usepackage{verbatim}
%\usepackage{mathbbol}
\usepackage[textsize=scriptsize]{todonotes}
\usepackage{tikz-cd}
\usepackage{etoolbox}
\usepackage{etex}
\usepackage{wasysym}
\usepackage[T1]{fontenc}
\usepackage{chemarr}
\usepackage{amssymb}
\usepackage[leqno]{amsmath}
\usepackage{comment}
\usepackage{mathtools}
\usepackage{rotating}
\usepackage{wrapfig}
\usepackage{outlines}
\usepackage{graphicx}
\usepackage{scalerel}
% \def\DevnagVersion{2.17}
% \usepackage{devanagari}
\usepackage{bbm}
\usepackage{multicol}

\usepackage{float} 
\usepackage{amsthm}
\usepackage[all,arc]{xy}
\usepackage{stackrel}


%\DeclareSymbolFontAlphabet{\amsmathbb}{AMSb}

%% ------------hood's sseq package----------------
\usepackage{spectralsequences}
%\usepackage[draft]{spectralsequences}

% Make widetilde work in spectral sequences.
\let\oldwidetilde\widetilde
\protected\def\widetilde{\oldwidetilde}

%%% Code from Hood to fix a positioning error with his spectralsequences package.
%% \usepackage{etoolbox}
%% \makeatletter
%% \patchcmd\endsseqpage{\path (0,-\sseq@yoffset) node}{\node}{}{\failed}
%% \makeatother


%% -----------------------------------------------
%  {\def\RSthmtxt{theorem~}\newref{thm}{name = \RSthmtxt}}

\newcommand{\corollaryname}{Corollary}
\newcommand{\definitionname}{Definition}
\newcommand{\examplename}{Example}
\newcommand{\lemmaname}{Lemma}
\newcommand{\propositionname}{Proposition}
\newcommand{\remarkname}{Remark}
\newcommand{\theoremname}{Theorem}

\newtheorem{theorem}{Theorem}
\renewcommand*{\thetheorem}{\Alph{theorem}}
%-----------





\usepackage{hyperref}
\usepackage{cleveref}

\hypersetup{
   colorlinks,
   linkcolor={red},
   citecolor={green!30!black},
   urlcolor={blue}
}

\usepackage{tikz}
\usetikzlibrary{matrix,arrows,decorations}
\usepackage{tikz-cd}

\usepackage{adjustbox}

\let\oldtocsection=\tocsection
 
\let\oldtocsubsection=\tocsubsection
 
\let\oldtocsubsubsection=\tocsubsubsection
 
\renewcommand{\tocsection}[2]{\hspace{0em}\oldtocsection{#1}{#2}}
\renewcommand{\tocsubsection}[2]{\hspace{1em}\oldtocsubsection{#1}{#2}}
\renewcommand{\tocsubsubsection}[2]{\hspace{2em}\oldtocsubsubsection{#1}{#2}}

% \numberwithin{section}{chapter}
% \numberwithin{subsection}{section}

\makeatletter
\newcommand{\leqnomode}{\tagsleft@true}
\newcommand{\reqnomode}{\tagsleft@false}
\makeatother


%---------------------------theorems-----------------------------------%

\theoremstyle{definition}

\newtheorem{nul}{}[section]
\newtheorem{dfn}[nul]{Definition}
\newtheorem{axm}[nul]{Axiom}
\newtheorem{rmk}[nul]{Remark}
\newtheorem{term}[nul]{Terminology}
\newtheorem{cnstr}[nul]{Construction}
\newtheorem{cnv}[nul]{Convention}
\newtheorem{ntn}[nul]{Notation}
\newtheorem{exm}[nul]{Example}
\newtheorem{obs}[nul]{Observation}
\newtheorem{ctrexm}[nul]{Counterexample}
\newtheorem{rec}[nul]{Recollection}
\newtheorem{exr}[nul]{Exercise}
\newtheorem{wrn}[nul]{Warning}
\newtheorem{qst}[nul]{Question}
\newtheorem{prb}[nul]{Problem}
\newtheorem{ass}[nul]{Assumption}
\newtheorem{hyp}[nul]{Hypothesis}
\newtheorem{warn}[nul]{Warning}

\newtheorem*{warn*}{Warning}
\newtheorem*{dfn*}{Definition}
\newtheorem*{axm*}{Axiom}
\newtheorem*{ntn*}{Notation}
\newtheorem*{exm*}{Example}
\newtheorem*{exr*}{Exercise}
\newtheorem*{int*}{Intuition}
\newtheorem*{qst*}{Question}
\newtheorem*{rmk*}{Remark}
\newtheorem*{comp1}{Computation 1:}


\theoremstyle{plain}

\newtheorem{sch}[nul]{Scholium}
\newtheorem{claim}[nul]{Claim}
\newtheorem{thm}[nul]{Theorem}
\newtheorem{prop}[nul]{Proposition}
\newtheorem{summary}[nul]{Summary}
\newtheorem{lem}[nul]{Lemma}
\newtheorem{var}[nul]{Variant}
\newtheorem{sublem}{Lemma}[nul]
\newtheorem{por}[nul]{Porism}
\newtheorem{cnj}[nul]{Conjecture}
\newtheorem{cor}[nul]{Corollary}
\newtheorem{fact}[nul]{Fact}
\newtheorem*{thm*}{Theorem}
\newtheorem*{prop*}{Proposition}
\newtheorem*{cor*}{Corollary}
\newtheorem*{lem*}{Lemma}
\newtheorem*{cnj*}{Conjecture}

\newtheorem*{theoremA}{Theorem A}

\newtheorem{innercustomthm}{Theorem}
\newenvironment{customthm}[1]
{\renewcommand\theinnercustomthm{#1}\innercustomthm}
{\endinnercustomthm}


%-------------------------------macros------------------------------------%


%----------limits and colimits-------%
\DeclareMathOperator*{\colim}{colim}
\DeclareMathOperator*{\limit}{limit}

\DeclareMathOperator{\cofiber}{cofiber}
\DeclareMathOperator{\cof}{cof}
\DeclareMathOperator{\fiber}{fiber}
\DeclareMathOperator{\fib}{fib}

\DeclareMathOperator{\img}{Img}

\DeclareMathOperator{\coker}{coker}
% \DeclareMathOperator{\ker}{ker}
\DeclareMathOperator{\Tot}{Tot}

%-------------mapping----------------%
\DeclareMathOperator{\Hom}{Hom}
% \DeclareMathOperator{\hom}{Hom} 
\DeclareMathOperator{\Map}{Map}
\DeclareMathOperator{\TC}{TC}
\DeclareMathOperator{\TCP}{TCP}

\DeclareMathOperator{\End}{End} 
\DeclareMathOperator{\Aut}{Aut}

\DeclareMathOperator{\Ext}{Ext}
\DeclareMathOperator{\ext}{Ext}
\DeclareMathOperator{\Fun}{Fun}
\DeclareMathOperator{\uHom}{\underline{Hom}}

%-------------operators----------------%


\DeclareMathOperator{\Spec}{\mathrm{Spec}}
\DeclareMathOperator{\SP}{\mathbb{S}}
\DeclareMathOperator{\Spf}{\mathrm{Spf}}
\DeclareMathOperator{\gl1}{gl_1}
\DeclareMathOperator{\sl1}{sl^E_1}

\def\spic{\mathfrak{pic}}
\def\spicp{\mathfrak{pic}[p]}
\DeclareMathOperator{\pic}{pic}
\DeclareMathOperator{\picc}{pic}%\pic is bad in tikzcd?
\DeclareMathOperator{\Pic}{Pic}
\DeclareMathOperator{\THH}{\mathrm{THH}}
\DeclareMathOperator{\TR}{\mathrm{TR}}
\DeclareMathOperator{\TP}{\mathrm{TP}}
\DeclareMathOperator{\BU}{\mathrm{BU}}
\DeclareMathOperator{\rB}{\mathrm{B}}
\DeclareMathOperator{\triv}{\mathrm{triv}}
\DeclareMathOperator{\Tr}{\mathrm{Tr}}
\DeclareMathOperator{\Res}{\mathrm{Res}}
\DeclareMathOperator{\Uloc}{\mathrm{Uloc}}
\DeclareMathOperator{\im}{im}
\DeclareMathOperator{\sBar}{\mathrm{Bar}}
\DeclareMathOperator{\Mod}{Mod}
\DeclareMathOperator{\Bimod}{Bimod}
\DeclareMathOperator{\CycSp}{CycSp}
\DeclareMathOperator{\Def}{Def}
\DeclareMathOperator{\Sub}{Sub}
\def\Lnp{L_{\overline{n}}}



\def\ct{\mathrm{ct}}
\def\Re{\mathrm{Re}}
\def\Pr{\mathrm{Pr}}
\def\PrL{\mathrm{Pr}^{\mathrm{L}}}
\def\st{\mathrm{st}}
\def\rig{\mathrm{rig}}

\def\Sym{\mathrm{Sym}}
\def\Ind{\mathrm{Ind}}
\def\Nm{\mathrm{Nm}}
\def\tr{\mathrm{tr}}
\def\height{\mathrm{height}}
\def\can{\mathrm{can}}
\def\len{\mathrm{len}}
\def\supp{\mathrm{supp}}
\newcommand{\ev}[0]{{\mathrm{ev}}}
%\newcommand{\ev}[2]{{\mathrm{ev}_{#2}^{#1}}}
\newcommand{\evo}[2]{{\overline{\mathrm{ev}}_{#2}^{#1}}}
\newcommand{\qin}{\quad\in\quad}%

%-------------objects----------------%
\def\A{\mathbb{A}}
\def\C{\mathbb{C}}
\def\E{\mathbb{E}}
\def\F{\mathbb{F}}
\def\G{\mathbb{G}}
\def\H{\mathbb{H}}
\def\N{\mathbb{N}}
\def\P{\mathbb{P}}
\def\Q{\mathbb{Q}}
\def\R{\mathbb{R}}
\def\Ss{\mathbb{S}}
\def\T{\mathbb{T}}
\def\W{\mathbb{W}}
\def\Z{\mathbb{Z}}
\def\cC{\mathcal{C}}
\def\CC{\mathcal{C}}
% \def\CCdef{\mathcal{C}_{\mathrm{def}}}
\def\DD{\mathcal{D}}
\def\AA{\mathcal{A}}
\def\sa{\mathcal{A}}
\def\EE{\mathcal{E}}
\def\FF{\mathbb{F}}
\def\D{\mathcal{D}}
\def\mE{\mathcal{E}}
\def\J{\mathcal{J}}
\def\cO{\mathcal{O}}
\def\cP{\mathcal{P}}
\def\X{\mathcal{X}}
\def\GL{\mathrm{GL}}
\def\Sq{\mathrm{Sq}}
\def\hR{\widehat{\mathcal{R}}}
\def\hRp{\widehat{\mathcal{R}}_{\F_p}}
\DeclareMathOperator{\CAlg}{CAlg}
\def\CAlgw{\mathrm{CAlg}^{\wedge}}
\def\Modw{\mathrm{Mod}^{\wedge}}



\def\one{\mathbbm{1}}
\def\pt{\mathrm{pt}}

\def\Ft{\mathbb{F}_2} % several variants of F_p
\def\Zt{\mathbb{Z}_2}
\def\MFt{\mathrm{M}\mathbb{F}_2}
\def\MZt{\mathrm{M}\mathbb{Z}_{2}}
\def\MFp{\mathrm{M}\mathbb{F}_p}
\def\uFt{\underline{\mathbb{F}}_2}
\def\uZ{\underline{\mathbb{Z}}}
\def\uZt{\underline{\mathbb{Z}}_{2}}
\def\Fp{\mathbb{F}_p}
\def\Fpbar{\overline{\mathbb{F}}_p}

\def\ko{\mathrm{ko}} % several variants of K-theory
\def\ku{\mathrm{ku}}
\def\KO{\mathrm{KO}}
\def\KU{\mathrm{KU}}
\def\kgl{\mathrm{kgl}}
\def\KGL{\mathrm{KGL}}
\def\kq{\mathrm{kq}}
\def\KQ{\mathrm{KQ}}

\def\tmf{\mathrm{tmf}} % several variants of tmf
\def\Tmf{\mathrm{Tmf}}
\def\TMF{\mathrm{TMF}}

\def\MU{\mathrm{MU}} % several variants of MU
\def\MUP{\mathrm{MUP}}
\def\BP{\mathrm{BP}}
\def\BPn{\mathrm{BP}\langle n \rangle}
\def\map{\mathrm{map}}
\def\bu{\mathrm{bu}}
\def\MGL{\mathrm{MGL}}
\def\MUR{\mathrm{MU}_{\mathbb{R}}}
\def\BPR{\mathrm{BP}_{\mathbb{R}}}

\def\Oo{\mathrm{O}} % A bunch of misc Thom spectra
\def\BO{\mathrm{BO}}
\def\MO{\mathrm{MO}}
\def\SO{\mathrm{SO}}
\def\BSO{\mathrm{BSO}}
\def\Spin{\mathrm{Spin}}
\def\BSpin{\mathrm{BSpin}}
\def\MSpin{\mathrm{MSpin}}
\def\Str{\mathrm{String}}
\def\BStr{\mathrm{BString}}
\def\MStr{\mathrm{MString}}
\def\BOn{\mathrm{BO \langle n \rangle}}
\def\MOn{\mathrm{MO \langle n \rangle}}

\def\Mfg{\mathcal{M}_{\mathrm{fg}}}

\def\RP{\R\mathrm{P}}
\def\CP{\C\mathrm{P}}

\def\m{\mathfrak{m}}
\def\modm{/\!\!/\m}
\def\mm{/\!\!/}
\def\ll{[\![}
\def\rr{]\!]}
\def\llp{(\!(}
\def\rrp{)\!)}

\newcommand{\SPp}{\SP_p}
%% ----- motivic stuff ------

% \def\DM{\mathrm{DM}}
% \def\nuc{\nu_{\mathbb{C}}}
% \def\nur{\nu_{\mathbb{R}}}

% \def\at{\scaleto{\mathrm{\bf \hspace{-1.5pt}A\hspace{-1.8pt}T}}{3.75pt}}
% \def\att{\scaleto{\mathrm{\bf A\hspace{-1.8pt}T}}{3.75pt}}
% \def\gceff{\scaleto{\mathrm{\bf A\hspace{-1.8pt}T}}{3.75pt}\mathrm{-eff}}
% \def\eff{\mathrm{eff}}
% \def\Sm{\textbf{Sm}}
% \def\cell{\mathrm{cell}}
% \def\res{\mathrm{res}}
% \def\triv{\mathrm{triv}}

% \def\MR{\mathrm{Mot}_{\mathbb{R}}}
% \def\MC{\mathrm{Mot}_{\mathbb{C}}}
% \def\b{\mathrm{Be}}
% \def\H{\mathrm{H}}


%-------------categories----------------%
\usepackage{babel}

\usepackage{cjhebrew}
\DeclareFontFamily{U}{rcjhbltx}{}
\DeclareFontShape{U}{rcjhbltx}{m}{n}{<->s*[1.2]rcjhbltx}{}
\DeclareSymbolFont{hebrewletters}{U}{rcjhbltx}{m}{n}
\DeclareMathSymbol{\pretsadi}{\mathord}{hebrewletters}{118}

\makeatother
%\newcommand{\comment}[1]{{\color{red}{\bf [#1]}}}
\newcommand{\cocone}{\mathbin{\rotatebox[origin=c]{90}{$\triangle$}}}
\newcommand{\cone}{\mathbin{\rotatebox[origin=c]{-90}{$\triangle$}}}
\newcommand{\es}{\operatorname{{\small \normalfont{\text{\O}}}}}
\newcommand{\iso}{\xrightarrow{\,\smash{\raisebox{-0.5ex}{\ensuremath{\scriptstyle\sim}}}\,}}
%\DeclareMathOperator{\supp}{supp}

\newcommand{\Yon}{\text{\usefont{U}{min}{m}{n}\symbol{'110}}}
% bookmarks=true,bookmarksnumbered=false,bookmarksopen=false,
% breaklinks=false,pdfborder={0 0 0},pdfborderstyle={},backref=false,colorlinks=true]
% {hyperref}
%\newcommand{\mdef}[1]{\textcolor{DefColor}{#1}} 
%Tobi: Not a big fan of this color for defined terms, so I removed it for now. revert if you prefer the previous version.
% \newcommand{\mdef}[1]{{#1}}
%\newcommand{\tdef}[1]{\textit{\mdef{#1}}} 

\newcommand{\dayc}{\mathrm{Day}}
\newcommand{\Perf}{\mathrm{Perf}_{\F_p}}
\newcommand{\canf}{\can^{\flat}}
\newcommand{\Tod}{\mathcal{T}}
%\newcommand{\can}{\mathrm{can}}
%\newcommand{\triv}{\mathrm{triv}}
\newcommand{\TCm}{\TC^{-}}
\newcommand{\eqlzr}{\bullet  \rightrightarrows  \bullet }
\newcommand{\pflat}{\pi_0^{\flat}}
\newcommand{\pflate}{\widetilde{\pi_0^{{\flat}}}}
\newcommand{\tcpflat}{\overline{\pi_0^{{\flat}}}}
\def\Alg{\mathrm{Alg}}
\def\CAlg{\mathrm{CAlg}}
\def\CRing{\mathrm{CRing}}
\def\CMon{\mathrm{CMon}}
\def\CAlgh{\mathrm{CRing}}
\def\Mod{\mathrm{Mod}}
\def\heart{\heartsuit}
\def\Modh{\mathrm{Mod}^{\heartsuit}}
\def\LMod{\mathrm{LMod}}
\def\Ab{\mathrm{Ab}}
\def\Abh{\mathrm{Ab}^{\heartsuit}}
\def\Sp{\mathrm{Sp}}
\def\LnfSp{L_n^f\mathrm{Sp}}
\def\Spaces{\mathcal{S}}
\def\Spc{\mathcal{S}}
\def\Set{\mathrm{Set}}
\def\Syn{\mathrm{Syn}}
\def\QCoh{\mathrm{QCoh}}
\def\IndCoh{\mathrm{IndCoh}}
\def\Groth{\mathrm{Groth}}

\def\BSL{\mathrm{BSL}}
\def\Top{\mathrm{Top}}
\def\CHaus{\mathrm{CHaus}}
\newcommand{\ezrtoend}{\mathfrak{R}}

%% -----------motivic stuff----------------

% \def\SHCa{\mathcal{SH}(\mathbb{C})^{\scaleto{\mathrm{\bf \hspace{-1.5pt}A}}{3.75pt}}}
% \def\SHCat{\mathcal{SH}(\mathbb{C})^{\scaleto{\mathrm{\bf \hspace{-1.5pt}A\hspace{-1.8pt}T}}{3.75pt}}_{ip}}
% \def\SHk{\mathcal{SH}(k)}
% \def\SHka{\mathcal{SH}(k)^{\scaleto{\mathrm{\bf \hspace{-1.5pt}A}}{3.75pt}}}
% \def\SHkt{\mathcal{SH}(k)^{\scaleto{\mathrm{\bf \hspace{-1.5pt}T}}{3.75pt}}}
% \def\SHkat{\mathcal{SH}(k)^{\scaleto{\mathrm{\bf \hspace{-1.5pt}A\hspace{-1.8pt}T}}{3.75pt}}}
% \def\SHRa{\mathcal{SH}(\mathbb{R})^{\scaleto{\mathrm{\bf \hspace{-1.5pt}A}}{3.75pt}}}
% \def\SHRt{\mathcal{SH}(\mathbb{R})^{\scaleto{\mathrm{\bf \hspace{-1.5pt}T}}{3.75pt}}}
% \def\SHRat{\mathcal{SH}(\mathbb{R})^{\scaleto{\mathrm{\bf \hspace{-1.5pt}A\hspace{-1.8pt}T}}{3.75pt}}_{i2}}
% \def\SHRatch{\mathcal{SH}(\mathbb{R})^{\scaleto{\mathrm{\bf \hspace{-1.5pt}A\hspace{-1.8pt}T}}{3.75pt}, \mathrm{C}-\heartsuit}_{i2}}
% \newcommand{\SHR}{\mathcal{SH}(\mathbb{R})}
% \newcommand{\SHC}{\mathcal{SH}(\mathbb{C})}
% \newcommand{\SH}{\mathcal{SH}}
% \newcommand{\SHgc}{\mathcal{SH}(\mathbb{R})^{\scaleto{\mathrm{\bf \hspace{-1.5pt}A\hspace{-1.8pt}T}}{3.75pt}}_{i2}}





%-------------adjectives---------------%
\def\slice{\mathrm{slice}}
\def\cb{\mathrm{cb}}
\def\diag{\mathrm{Diag}}
\def\even{\mathrm{even}}
\def\twist{\mathrm{tw}}
\def\Fin{\mathrm{Fin}}
\def\op{\mathrm{op}}
\def\Gr{\mathrm{Gr}}
\def\gr{\mathrm{gr}}
\def\Fil{\mathrm{Fil}}
\def\perf{\sharp}
\def\LT{\mathrm{LT}}
\def\cons{\mathrm{cons}}


\newcommand{\pushout}{\arrow[ul, phantom, "\ulcorner", very near start]}
\newcommand{\pullback}{\arrow[dr, phantom, "\lrcorner", very near start]}


%-------------exotic letters---------------%

% \def\dtau{\text{\scriptsize {\dn tO}}}
% \def\ta{\text{\footnotesize {\dn t}}}
\def\kappabar{\overline{\kappa}}
%\DeclareSymbolFont{extraitalic}      {U}{zavm}{m}{it}
%\DeclareMathSymbol{\Qoppa}{\mathord}{extraitalic}{161}
%\DeclareMathSymbol{\qoppa}{\mathord}{extraitalic}{161}
%\DeclareMathSymbol{\sampi}{\mathord}{extraitalic}{166}
\newcommand{\pointL}[2]{\one_{#1}[#2]}
\newcommand{\pointR}[1]{\Omega^{\infty}_{#1}}
\newcommand{\ldbl}{(\!(}
\newcommand{\QQ}{\mathbb{Q}}
\newcommand{\ZZ}{\mathbb{Z}}
\newcommand{\NN}{\mathbb{N}}
\newcommand{\val}{\nu}
\newcommand{\arc}{\mathrm{arc}}
\newcommand{\HRing}[2]{#1\ll t^{#2} \rr}
\newcommand{\QAlg}{\mathrm{Ring}_{\QQ}}
\newcommand{\rdbl}{)\!)}
\newcommand{\mate}[1]{{#1}^{\tiny\gemini}}
\newcommand{\wt}{\widetilde}
\newcommand{\MF}{\mathrm{MF}}
\newcommand{\cycl}{^{\wedge}_{\mathrm{cyc}}}
\newcommand{\bp}{\mathrm{bP}}
\newcommand{\Orb}{\mathrm{Orb}}
\newcommand{\Prig}{\mathrm{Pr^{rig}}}
\newcommand{\grmot}{\gr_{\mathrm{mot}/\mathrm{MU}}}
\newcommand{\dualz}{\diamondsuit}
\newcommand{\dual}{\vee}
%\newcommand{\Mod}{\mathrm{Mod}}
\newcommand{\qind}{\quad\in\quad}%
\global\long\def\oto#1{\xrightarrow{#1}}%
\definecolor{DefColor}{rgb}{0.6,0.15,0.25}
\newcommand{\Setc}{\mathrm{Set}}
\newcommand{\mdef}[1]{\textcolor{DefColor}{#1}} 
\newcommand{\tdef}[1]{\textit{\mdef{#1}}}
\newcommand{\deff}[1]{\tdef{#1}}
\newcommand{\idem}{\mathrm{idem}}
\newcommand{\Spd}{\mathrm{Spd}}
\def\id{\mathrm{id}}
\def\Id{\mathrm{Id}}
\newcommand{\NS}{\mathrm{NS}}
\newcommand{\NSP}{\mathrm{NS}^{\Pi}}
%----------------------------------------------------------------------%
\newcommand{\CSpec}{\mathrm{Spec}^{\mathrm{cons}}_{T(n)}}
% triangles
\newcommand\xqed[1]{%
  \leavevmode\unskip\penalty9999 \hbox{}\nobreak\hfill
  \quad\hbox{#1}}
\newcommand\tqed{\xqed{$\triangleleft$}}



\newcommand{\Npm}{t^{\pm 1 /p^{\infty}}}

\newcommand{\Np}{t^{1/p^{\infty}}}
\newcommand{\Npp}{t^{1/p^{\infty}}, s^{1/p^{\infty}}}
\def\Perf{\mathrm{Perf}}
\newcommand{\noloc}{\;\mathord{:}\,}

\usepackage{amssymb,graphicx}

\newcommand{\LCF}[1]{C^0(#1;\F_p)}
\newcommand{\on}{\mathbbm{1}}
\newcommand{\cyc}[1]{\Z/p^{#1}\Z}
%\newcommand{\Map}{\mathmrm{Map}}
%\newcommand{\Spec}{\mathrm{Spec}}
%\newcommand{\o1}{\mathbbm{1}}
%\newcommand{\cC1}{\mathcal{C}}
\newcommand{\Zp}{\mathbb{Z}_p}






\newcommand{\bbGamma}{{\mathpalette\makebbGamma\relax}}
\newcommand{\makebbGamma}[2]{%
  \raisebox{\depth}{\scalebox{1}[-1]{$\mathsurround=0pt#1\mathbb{L}$}}%
}



%%%Local Variables:
%%% mode: latex
%%% TeX-master: "Main"
%%% eval: (visual-line-mode t)
%%% eval: (auto-fill-mode 0)
%%% End:

\usepackage[margin=1.5in]{geometry}

\newtoggle{draft}
% NB: toggle this to hide comments
%\togglefalse{draft}
\toggletrue{draft}

\iftoggle{draft}{
%\usepackage{showkeys}
\newcommand{\atodo}[1]{\todo[color=blue!40]{#1}}
\newcommand{\ttodo}[1]{\todo[color=green!40]{#1}}
\newcommand{\tcomment}[1]{\todo[inline,color=green!40]{#1}}
\newcommand{\NB}[1]{\todo[color=gray!40]{#1}}
\newcommand{\TODO}[1]{\todo[color=red]{#1}}
}{ % else
\newcommand{\NB}[1]{}
\newcommand{\TODO}[1]{}
\renewcommand{\todo}[1]{}
\renewcommand{\todo}[1]{}
}

\title{Exercise sheet 2}

%% \author{Robert Burklund}
%% \address{Department of Mathematical Sciences, University of Copenhagen, Denmark}
%% \email{rb@math.ku.dk}

\begin{document}
\maketitle

\section*{\bf Exercise 1: Advanced shuffling}

In lecture we discussed the shuffling formulas
\begin{itemize}
\item $\langle a,b,cd \rangle \subseteq \langle a,bc,d \rangle$,
\item $\langle a,b,c \rangle d = -a \langle b,c,d \rangle$ and
\item $\langle a,b,c \rangle + \langle b,c,a \rangle + \langle c,a,b \rangle = 0$
\end{itemize}
and the situations in which they apply.
In this exercise you will prove two additional shuffling formulas.

Let $\CC$ be a stable category.
Let
\[ U \xrightarrow{a} V \xrightarrow{b} W \xrightarrow{c} X \xrightarrow{d} Y \]
be a sequence of four composable maps in $\CC$ and
suppose that all $ab$ and $bc$ are nullhomotopic.

\begin{enumerate}
\item[(a)] Show that
  \[ \langle a,b,c \rangle d \subseteq \langle a,b,cd \rangle. \]
\item[(b)] Identify which nullhomotopies one should use to obtain a equality.
\end{enumerate}

Let
\[ U \xrightarrow{a} V \xrightarrow{b} W \xrightarrow{c} X \xrightarrow{d} Y \xrightarrow{e} Z \]
be a sequence of five composable maps in $\CC$ and
let $e_2 : bc \leftrightarrow 0$ and $e_3 : cd \leftrightarrow 0 $ be nullhomotopies.
Suppose further that all of the following maps are nullhomotopic
\[ ab,\quad de,\quad a  \langle b,{}^{e_2} c,{}^{e_3} d \rangle,\quad \langle b,{}^{e_2} c,{}^{e_3} d \rangle e. \]

Consider the following expression:
\begin{align}
  \langle \langle a, b, c \rangle, d, e \rangle \pm \langle a, \langle b, c, d \rangle, e \rangle \pm \langle a, b, \langle c, d, e \rangle \rangle
\end{align}

\begin{enumerate}
\item[(c)] Confirm that the expression in (1) is defined,
  adding in the omitted suspensions as necessary.
\item[(d)] Determine an orientation of the signs so that (1) contains zero.
\item[(e)] Identify a collection of nullhomotopies so that the expression is exactly zero.
\end{enumerate}


\section*{\bf Background}

The first $13$ stable homotopy groups of the $2$-local sphere $\Ss_{(2)}$ can be computed using only the Serre spectral sequence. 
These groups (and the ring structure) are given in the table below:

\begin{center}
  \begin{tabular}{|c|c|c|c|}\hline
    stem & group & generator & relations \\\hline\hline
    0 & $\Z$ & $1$ & \\\hline
    1 & $\Z/2$ & $\eta$ & \\\hline
    2 & $\Z/2$ & $\eta^2$ & \\\hline
    3 & $\Z/8$ & $\nu$ & $\eta^3 = 4\nu$ \\\hline
    4 & $0$ & & \\\hline
    5 & $0$ & & \\\hline
    6 & $\Z/2$ & $\nu^2$ & \\\hline
    7 & $\Z/16$ & $\sigma$ & \\\hline
    8 & $\Z/2 \oplus \Z/2$ & $\eta\sigma$, $\epsilon$ & \\\hline
    9 & $\Z/2 \oplus \Z/2 \oplus \Z/2$ & $\eta^2\sigma$, $\eta\epsilon$, $\mu_9$ & $\nu^3 = \eta^2\sigma + \eta\epsilon$ \\\hline
    10 & $\Z/2$ & $\eta\mu_9$ & \\\hline
    11 & $\Z/8$ & $\zeta$ & $\eta^2\mu_9 = 4\zeta$ \\\hline
    12 & $0$ & & \\\hline
    13 & $0$& & \\\hline
  \end{tabular}
\end{center}

You will want to use this information as you do the next four exercises.

%% {\bf Question 2:} Evaluating Toda brackets \hfill
\section*{\bf Exercise 2: Evaluating Toda brackets}

Toda proved the following formula\footnote{This formula comes from the theory of power operations on $\Ss$ and we will return to this subject at the end of the course.} for Toda brackets of elements in $\pi_*\Ss$:
Given $\alpha, \beta \in \pi_*\Ss$ such that $\alpha\beta = 0$
\begin{itemize}
\item If $|\alpha| = k$ is even (or $k$ is odd and $2\alpha =0$),
  there is an element $Q_1(\alpha) \in \pi_{2k+1}\Ss$ such that
  \[ \langle \alpha, \beta, \alpha \rangle = Q_1(\alpha) \beta + \alpha \blacksquare. \]
  Here $\blacksquare$ is an ideterminate used for keeping track of indeterminacy.
\item If $|\alpha| = k$ is odd, then   
  \[ \langle \alpha, \beta, \alpha \rangle \quad\text{ and }\quad \langle \beta, \alpha, 2\alpha \rangle. \]
  have non-empty intersection.
\end{itemize}
For elements in even degree $Q_1(-)$ is a (not necessarily additive) operation of signature
\[ \pi_k\Ss \to \pi_{2k+1}\Ss. \]
In fact, for $\alpha_1, \alpha_2 \in \pi_k\Ss$ with $k$ even we have
\[ Q_1(\alpha_1 + \alpha_2) = Q_1(\alpha_1) + Q_1(\alpha_2) + \eta \alpha_1 \alpha_2 \]
and for $\alpha_1 \in \pi_{k_1}\Ss$ and $\alpha_2 \in \pi_{k_2}\Ss$ with $k_1,k_2$ even we have
\[ Q_1(\alpha_1 \alpha_2) = \alpha_1^2 Q_1(\alpha_2) + Q_1(\alpha_1) \alpha_2^2 \]

Using the table above, shuffling rules and Toda's formula evaluate the following Toda brackets.
Remember to keep indeterminacy in mind.

\begin{enumerate}
\item[(0)] Compute $Q_1(n)$ for each $n \in \Z \cong \pi_0\Ss$.
\item[(a)] $\langle 2, \eta, 2 \rangle$
\item[(b)] $\langle \eta, 2, \eta \rangle$
\item[(c)] $\langle 2, \eta, \nu \rangle$
\item[(d)] $\langle \eta, \nu, \eta \rangle$
\item[(e)] $\langle \nu, \eta, \nu \rangle$
\item[(f)] $\langle \eta^2, \eta^2, \eta^2 \rangle$
\item[(g)] $\langle \eta, 2, \nu^2 \rangle$
\item[(h)] $\langle \mu_9, 2, \eta \rangle$
\end{enumerate}

\section{\bf Exercise 3: Cells structures}

Like spaces, spectra can be given skeletal filtrations.

\begin{dfn}
  A skeletal filtration on $X$ is a sequence 
  \[ \cdots \to X_{-2} \to X_{-1} \to X_0 \to X_1 \to X_2 \to \cdots \]
  such that
  \begin{enumerate}
  \item $\cof(X_{i-1} \to X_i)$ is a sum of copies of $\Ss^i$,
  \item $0 \cong X_{-\infty} \coloneqq \lim X_i$ and
  \item $X \cong X_\infty \coloneqq \colim X_i$.
  \end{enumerate}
\end{dfn}

Alternatively, we can rotate the cofiber sequences in a skeletal filtration and view
the pushout
\[ \begin{tikzcd}
  \oplus \Ss^{i-1} \ar[r] \ar[d] & X_{i-1} \ar[d] \\
  0 \ar[r] & X_{i}
\end{tikzcd} \]
as specifying a collection of cell attachments that build $X_i$ from $X_{i-1}$.\footnote{Recall that discs are contractible, which is why the bottom left is zero.}


\begin{enumerate}
\item[(a)] Prove that every $2$-local finite spectrum $X$ can be given a skeletal filtration with exactly $\mathrm{dim}_{\F_2}(H_i(X;\F_2))$ many $i$-cells. We call a choice of such a skeletal filtration a minimal skeletal filtration. (Hint: use the Hurewicz theorem)
\item[(b)] Extend the results of part (a) to spectra which are of finite type.\footnote{$X$ is of finite type if it is bounded below and $H_i(X;\Z_{(2)})$ is finitely generated for all $i$.}
\end{enumerate}

The homotopy and homology rings of the spectrum $\F_2$ representing cohomology with $\F_2$ coefficients are given by
\[ \pi_*\F_2 \cong \begin{cases} \F_2 & *=0 \\ 0 & *\neq 0 \end{cases} \qquad\qquad H_*(\F_2; \F_2) \coloneqq \pi_*(\F_2 \otimes \F_2) \cong \F_2[\zeta_1, \zeta_2, \dots] \]
where $|\zeta_i| = 2^{i} - 1$.

\begin{enumerate}
\item[(c)] Determine the attaching maps in the minimal $0$-skeleton of $\F_2$.
\item[(d)] Determine the attaching maps in the minimal $1$-skeleton of $\F_2$.
\item[(e)] Determine the attaching maps in the minimal $2$-skeleton of $\F_2$.
\item[(f)] Determine the attaching maps in the minimal $3$-skeleton of $\F_2$.
\end{enumerate}

The homotopy and homology rings of the spectrum $\Z_{(2)}$ representing cohomology with $\Z_{(2)}$ coefficients are given by
\[ \pi_*\F_2 \cong \begin{cases} \Z_{(2)} & *=0 \\ 0 & *\neq 0 \end{cases} \qquad\qquad H_*(\Z_{(2)}; \F_2) \coloneqq \pi_*(\Z_{(2)} \otimes \F_2) \cong \F_2[\zeta_1^2, \zeta_2, \dots] \]
where $|\zeta_1^2| = 2$ and $|\zeta_i| = 2^{i} - 1$ for $i \geq 2$.
The name of the class $\zeta_1^2$ comes from the fact that the reduction mod $2$ map
\[ \Z_{(2)} \to \F_2 \]
induces the natural injective map of polynomial algebras
\[ \F_2[\zeta_1^2, \zeta_2, \dots] \to \F_2[\zeta_1, \zeta_2, \dots] \]
on $\F_2$-homology.

\begin{enumerate}
\item[(g)] Determine the attaching maps in the minimal $2$-skeleton of $\Z_{(2)}$.
\item[(h)] Determine the attaching maps in the minimal $3$-skeleton of $\Z_{(2)}$.
\end{enumerate}


\section{\bf Exercise 4: Characterizing $\mathrm{KU}$}

$\mathrm{KU}$ is an $\E_\infty$-algebra in $\Sp$ whose associated cohomology theory
classifies stable complex vector bundles.
The ring spectrum $\mathrm{KU}_{(2)}$ satisfies the following properties

\begin{enumerate}
\item $\KU_{(2)}$ is $2$-periodic.
  This means that there is an invertible class $\beta$ in $\pi_2\KU_{(2)}$.
\item The connective cover of $\KU_{(2)}$ is typically called $\ku_{(2)}$.
  The $\E_\infty$-algebra $\ku_{(2)}$ is finite type and 
  its $\F_2$-homology is
  \[ H_*(\ku_{(2)}; \F_2) \coloneqq \pi_*(\F_2 \otimes \ku_{(2)}) \cong \F_2[\zeta_1^{2}, \zeta_2^{2}, \zeta_3, \zeta_4, \dots ] \]
  where
  $|\zeta_1^{2}| = 2$,
  $|\zeta_2^{2}| = 6$
  and $|\zeta_i| = 2^{i} - 1$ for $i \geq 3$.
\item There is a map of $\E_\infty$-algebras $\ku_{(2)} \to \Z_{(2)}$ which induces the natural inclusion
  \[ \F_2[\zeta_1^{2}, \zeta_2^{2}, \zeta_3, \zeta_4, \dots ] \to \F_2[\zeta_1^2, \zeta_2, \dots] \]
  on $\F_2$-homology.
\end{enumerate}


\begin{enumerate}
\item[(a)] Show that the localization of $\ku_{(2)}$ at $\beta$ is $\KU_{(2)}$.
\item[(b)] Determine the cells and attaching maps of a minimal 4 skeleton of $\ku_{(2)}$.
\item[(c)] Using (b), compute $\pi_*\ku_{(2)}$ for $*=0,1,2$.
\item[(d)] Determine the ring $\pi_*\KU_{(2)}$.
\item[(e)] Describe $\beta$ as a map into the $2$-skeleton of $\ku_{(2)}$.
\item[(f)] Conclude that properties (1), (2) and (3) above characterize $\KU_{(2)}$.
\item[(g)] How much can you weaken properties (1), (2) and (3) while still being able to prove (f)?
\end{enumerate}


%% \section{\bf Exercise 5: The Wood cofiber sequence}

%% $\mathrm{KO}$ is an $\E_\infty$-algebra in $\Sp$ whose associated cohomology theory
%% classifies stable vector bundles.
%% Complexification gives a map of $\E_\infty$-algebras
%% \[ \KO \to \KU. \]

%% homotopy groups of both

%% (a) determine what the map does on homotopy.


%% (b) Reduce computing cof eta homotopy to a single bracket.



%% (c)

%% The ring spectrum $\mathrm{KU}$ satisfies the following properties


\bibliographystyle{alpha}
\bibliography{bibliography}

\end{document}

%%% Local Variables:
%%% mode: latex
%%% eval: (visual-line-mode t)
%%% eval: (auto-fill-mode 0)
%%% End:
